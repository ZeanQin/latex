%% Start of Preamble

\documentclass[12pt]{article}
\usepackage{amsmath, amssymb, amsthm}
% This is the encoding for the document. 
% It can be omitted or changed to another encoding but utf-8 is recommended. 
% Unless you specifically need another encoding, or if you are unsure about it, add this line to the preamble.
\usepackage[utf8]{inputenc} 

\title{Sample \LaTeX File}
\author{Zean Qin \thanks{Nobody}}
\date{December 2019}

%% End of Preamble

\begin{document}

\maketitle
\tableofcontents

\section{Math Mode}
\subsection{Inline mode}
Inline mode is used to write formulas that are part of a text. To display math in inline mode, use any of the following, 

\begin{itemize}
    \item $\setminus$( ... $\setminus$)
    \item \$ ... \$
    \item $\setminus$begin\{math\} ... $\setminus$end\{math\}
\end{itemize}

\begin{verbatim}
In physics, the mass-energy equivalence is stated by 
the equation $E=mc^2$, discovered in 1905 by Albert Einstein.
\end{verbatim}

\subsection{Display mode}
Display mode is used to write expressions that are not part of a text or paragraph, and are therefore put on separate lines.

The displayed mode has two versions: numbered and unnumbered.

\begin{itemize}
    \item $\setminus$[ ... $\setminus$]
    \item $\setminus$begin\{displaymath\} ... $\setminus$end\{displaymath\}
    \item $\setminus$begin\{equation\} ... $\setminus$end\{equation\}
    \item \$\$ ... \$\$. This is discouraged as it can give inconsistent spacing, and may not work well with some math packages.
    
    
\end{itemize}

\section{Lists}
\subsection{Unordered lists}

Unordered lists are produced by the \textbf{itemize} environment. Each entry must be preceded by the control sequence $ \setminus $\textbf{itemize} as shown below.

\begin{verbatim}
\begin{itemize}
  \item The individual entries are indicated with a black dot, 
        a so-called bullet.
  \item The text in the entries may be of any length.
\end{itemize}
\end{verbatim}

By default the individual entries are indicated with a black dot, so-called bullet. The text in the entries may be of any length.

\subsection{Ordered lists}

Ordered list have the same syntax inside a different environment. We make ordered lists using the enumerate environment:

\begin{verbatim}
\begin{enumerate}
  \item This is the first entry in our list
  \item The list numbers increase with each entry we add
\end{enumerate}
\end{verbatim}

becomes 

\begin{enumerate}
  \item This is the first entry in our list
  \item The list numbers increase with each entry we add
\end{enumerate}

\end{document}